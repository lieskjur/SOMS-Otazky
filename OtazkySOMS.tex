\documentclass{article}
\usepackage[utf8]{inputenc}

% Basic packages
	\usepackage{amssymb}
	\usepackage{amsmath}
	\usepackage{graphicx}
	\usepackage[czech]{babel}
	\usepackage{natbib}

% Relevant packages
	\usepackage{bm}
	\usepackage{bbm}
	\usepackage{geometry}
	\usepackage{enumitem}
	\usepackage{listings}
	\usepackage{multicol,float}

% Document settings
	\geometry{a4paper,margin=15mm}

	\setlist[itemize]{nolistsep,noitemsep}
		
	\setlength{\parindent}{0pt}
	\setlength{\parskip}{1.5em}
	\renewcommand{\baselinestretch}{1.2}

	\setlength{\abovedisplayskip}{1.2em}
	\setlength{\belowdisplayskip}{1.2em}
	\renewcommand{\arraystretch}{1.2}

\begin{document}
	\section{13 Kalibrace}

	\begin{equation}
		\bm{f}(\bm{x},\bm{p}) = \bm{0}
	\end{equation}
	kde $\bm{x}$ jsou měřené veličiny a $\bm{p}$ kalibrační parametry

	\begin{equation}
		\bm{F}(\bm{X},\bm{p})
		=
		\bm{0} \doteq \bm{F}(\bm{X},\bm{\overline{p}})
		+
		\underbrace{\frac{\partial \bm{F}(\bm{X},\bm{\overline{p}})}{\partial p}}_{\bm{J}_p} \Delta\bm{p}
		\;\Rightarrow\;
		\Delta\bm{p} = \bm{J}_p^+ \bm{F}(\bm{X},\bm{\overline{p}})
	\end{equation}
	kde
	\begin{equation}
		\bm{F}
		=
		\begin{bmatrix}
			\bm{f}(\bm{x}_1,\bm{p}) = \bm{0} \\
			\vdots \\
			\bm{f}(\bm{x}_n,\bm{p}) = \bm{0}
		\end{bmatrix}
		\;,\quad 
		\bm{X}
		=
		\begin{bmatrix}
			\bm{x}_1 \\
			\vdots \\
			\bm{x}_n
		\end{bmatrix}
	\end{equation}

	Kalibrovatelnost je obcená schopnost kalibrovat mechanismus. Závisí na volbě kalibrovaných parametrů i kalibračních poloh. Značnou roli hraje podmíněnost matice $\bm{J}_p^T \bm{J}_p$.

	\section*{14 Rozšířená kalibrovatelnost}
	\begin{equation}
		\bm{f}(\bm{\overline{x}}+\bm{\hat{x}},\bm{\overline{p}}+\bm{\hat{p}}) = \bm{0}
	\end{equation}
	kde $\bm{\overline{x}}$ jsou měřené veličiny, $\bm{\hat{x}}$ opravy veličiny, $\bm{p}$ kalibrační parametry a $\bm{\hat{p}}$ opravy parametrů.

	Cílem je minimalizovat kritérium
	\begin{equation}
	\chi^2 = \sum_{i=1}^N \bm{\hat{x}}_i^T \bm{C}_x^{\frac{1}{2}} \bm{\hat{x}}_i + \bm{\hat{p}}^T \bm{C}_p^{\frac{1}{2}} \bm{\hat{p}} 
	\end{equation}
	Při splnění
	\begin{equation}
		\sum_{i=1}^N \bm{f}(\bm{\overline{x}}_i+\bm{\hat{x}}_i,\bm{\overline{p}}+\bm{\hat{p}}) = \bm{0}
	\end{equation}
	To zprostředkujeme zavedením Lagrangiánu
	\begin{equation}
		L = \chi^2 + \sum_{i=1}^N \lambda_i \bm{f}(\bm{\overline{x}}_i+\bm{\hat{x}}_i,\bm{\overline{p}}+\bm{\hat{p}})
	\end{equation}

	\section*{15 Beneš-Šika kalibrace}

	Přechod mezi následujícími osami lze popsat 4 parametry $\Rightarrow$

	\begin{align}
		&\underbrace{
			\bm{T}_x(x_0)\bm{T}_y(y_0)\bm{T}_z(z_0)\bm{T}_{\phi_x}(\phi_{x_0})\bm{T}_{\phi_y}(\phi_{y_0})\bm{T}_{\phi_z}(\phi_{z_0})
		}_{\text{Tracker-Základna}}
		\bm{T}_{\phi_z}(\phi_{12})
		\underbrace{
			\bm{T}_x(x_2)\bm{T}_z(z_2)\bm{T}_{\phi_x}(\phi_{x_2})\bm{T}_{\phi_z}(\phi_{z_2})
		}_{\text{Kalibrace na tělesu 2}}\\&
		\bm{T}_{\phi_x}(-\phi_{23})
		\underbrace{
			\bm{T}_x(x_3)\bm{T}_z(z_3)\bm{T}_{\phi_x}(\phi_{x_3})\bm{T}_{\phi_z}(\phi_{z_3})
		}_{\text{Kalibrace na tělesu 3}}
		\ldots
		\bm{r}_{N K} = \bm{r}_{0K}
	\end{align}
	(pls někdo doplňte zbytek rovnice a obrázek)

	Pro každou kalibrační polohu máme 3 rovnice a mechanismus má dohromady 27 kalibračních parametrů $\Rightarrow$ potřebujeme min. 9 kalibračních poloh.


\end{document}